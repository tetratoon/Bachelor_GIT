\documentclass[a4paper,11pt]{article} 
\begin{document} 
\title{Example 4} 
\author{My name} 
\date{January 5, 2011} 
\maketitle 
\section{What's this?} 
This is our 
second document. 
It contains two paragraphs. The first line of a paragraph will be 
indented, but not when it follows a heading. 
\section{Vorgaben}
Nachvollziehbarer Prozess durch LaTeX und GIT
Um den Prozess der Erstellung der Bachelorarbeit nachvollziehbar zu gestalten ist für die Verfassung der Arbeiten LaTeX vorgesehen. Der Fortschritt bei der Erstellung der Arbeit wird durch die Verwendung des GIT-Systems (https://gitlab.ima.fh-joanneum.at/) dokumentiert. 
Eine aktuelle LaTeX-Vorlage zur Erstellung der BAC-Arbeit wird über das GIT-System zur Verfügung gestellt. Um den kontinuierlichen Erstellungsprozess zu dokumentieren ist ab dem Beginn der Erstellung bis zur Abgabe der Arbeit mindestens alle zwei Wochen die aktuelle Version der Arbeit im GIT-System hochzuladen (commit) und ein aussagekräftiger Kommentar zum aktuellen Fortschritt anzugeben.

% Here's a comment. 
\end{document}
